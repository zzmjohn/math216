\documentclass[oneside, openany]{memoir}
\usepackage[leqno]{amsmath}
\usepackage{amsthm}
\usepackage{hyperref}
\usepackage{thmtools}
\usepackage{xspace}

\usepackage{tgpagella}
\usepackage[sc]{mathpazo}
\usepackage{microtype}

\chapterstyle{companion}

\swapnumbers
\declaretheoremstyle[headfont = \scshape, bodyfont = \normalfont, qed = \qedsymbol]{foag}
\declaretheorem[style = foag, numberwithin = section]{exercise}
\renewcommand{\theexercise}{\textbf{\arabic{chapter}.\arabic{section}.\Alph{exercise}}}

\newcommand\foreign[1]{\emph{#1}}
\newcommand\eg{\foreign{e.g.}}
\newcommand\Eg{\foreign{E.g.}}
\newcommand\ie{\foreign{i.e.}\xspace}
\newcommand\Ie{\foreign{I.e.}}
\newcommand\vs{\foreign{vs.~}}
\newcommand\cf{\foreign{cf.\xspace}}
\newcommand\resp{\foreign{resp.}}
\newcommand\etc{\foreign{etc.}}

\newcommand\category[1]{\mathsf{#1}}

\DeclareMathOperator\Mor{Mor}

\title{Solutions to Foundations of Algebraic Geometry}
\author{Pieter Belmans}

\begin{document}

\maketitle

\begin{abstract}
  These notes contain my solutions to Ravi Vakil's Foundations of Algebraic Geometry. For more information I refer you to the official website/blog (\url{http://math216.wordpress.com/}) and the page with the actual notes (\url{http://math.stanford.edu/~vakil/216blog/}).
  
  The solutions are provided as is. I don't claim these to be correct or well written (although I certainly intend them to be for my own benefit). If you encounter any flagrant mistakes, you can contact me by e-mail (\url{pieterbelmans@gmail.com}) or add a patch to the GitHub repository at \url{https://github.com/pbelmans/math216}. Please do so by the way, preferably by submitting patches.

  As the notes are still being written when I started writing up these solutions and there may be changes to the numbering system upcoming: I am using the June 27 version. If I am still interested in these exercises when a final version is posted, I might edit in possible changes.
\end{abstract}

\clearpage

\tableofcontents*


\chapter{Introduction}
There are no exercises in this chapter.


\part{Preliminaries}

\chapter{Some category theory}
\section{Motivation}

There are no exercises in this section.


\section{Categories and functors}

\begin{exercise}
  \begin{enumerate}
    \item The elements of the group(oid) correspond to the morphisms. As every morphism is an isomorphism, we can only compose morphisms on the same object. Now in case of a single object, all axioms for a group are satisfied, as isomorphisms lead to inverse elements.

    \item Take the category of a group and copy the unique object, together with all its morphisms. Voil\`a, a groupoid.

      A natural example of groupoids is the fundamental groupoid of a topological space. You cannot combine loops that have different base points.
  \end{enumerate}
\end{exercise}

\begin{exercise}
  By definition of \emph{invertible element of~$\Mor(A,A)$} the automorphisms form a group: we get composition and associativity from the category and the identity and inverse from our choice of elements.
  
  In case of~$\category{Sets}$ the automorphisms are the permutations of the set, \ie, bijections. In case of~$\category{Vec}_k$ the automorphisms are bijective linear self-maps.

  By conjugation, isomorphic objects have isomorphic automorphism groups.
\end{exercise}

\begin{exercise} % TODO I should probably do this
  Linear algebra exercise. I will do this one if I feel inspired.
\end{exercise}

\begin{exercise}
  A basis for a finite-dimensional vectorspace has a well-defined cardinality, defining its dimension. So the inverse functor~$\category{f.d.\ Vec}_k\to\mathcal{V}$ maps an~$n$\nobreakdash-dimensional vectorspace~$V$ to~$k^n$, while every linear map between finite-dimensional vectorspaces can be written against a choice of bases. As we were allowed to pick a basis for each vectorspace simultaneously, every linear map admits by linear algebra magic with tons of indices a representation as a matrix.
\end{exercise}


\section{Universal properties determine an object up to unique isomorphism}

\begin{exercise}
  \label{exercise:23a}
  Take~$A$ and~$B$ initial objects. As there exist (unique) morphisms~$A\to B$ and~$B\to A$, we can compose them, obtaining morphisms~$A\to A$ and~$B\to B$. But the identity is another candidate for this morphisms, so by uniqueness of the morphisms~$A$ and~$B$ are isomorphic.

  The proof for final objects is completely the same.
\end{exercise}

\begin{exercise}
  \begin{description}
    \item[$\category{Sets}$] The initial object is the empty set~$\emptyset$, the singleton~$\left\{ x \right\}$ is the final object (all singletons are isomorphic as stated before in~\autoref{exercise:23a}).
    \item[$\category{Rings}$] As the image of~$1\in\mathbb{Z}$ determines the entire ring morphism, the ring of integers~$\mathbb{Z}$ is the initial object. The final object is the trivial ring (in which~$0=1$).
    \item[$\category{Top}$] The initial object is the empty set~$\emptyset$ equipped with the topology consisting of~$1$~open set, the final object is the singleton equipped with the topology consisting of the empty set and the entire space.
  \end{description}

  The category subsets of a set and the category of open sets in a topological space are both \emph{bounded lattices}. Or, as there is either no morphism (if two sets are incomparable) or one morphism (if one set is contained in the other), we need to find objects that are either smaller or greater than all other objects. These are the empty set and the set~$X$.
\end{exercise}

\begin{exercise}
  Take~$s\in S$ a zerodisivor and let~$b\in A$ such that~$bs=0$. Now the image of~$b$ is equal to zero as
  \begin{equation}
    \frac{b}{1}=\frac{0}{1}\Longleftrightarrow s\left( b-0 \right)=0.
  \end{equation}

  Conversely, take~$a,b\in A$ and~$a\neq b$ such that their images are equal in the localization. That means there exists an~$s\in S$ such that~$s(a-b)=0$, so~$a-b\neq 0$ is a zerodivisor as~$0\notin S$.
\end{exercise}

\begin{exercise}
  \label{exercise:23d}
  The~$A$\nobreakdash-algebra~$S^{-1}A$ is a member of this category: an element~$s\in S$ is a unit as it is inverted by~$1/s$. It is furthermore initial among these algebras because the unique morphism~$\overline{\varphi}\colon S^{-1}A\to B$ is given by~$\overline{\varphi}(r/s)=\varphi(r)\varphi(s)^{-1}$ where~$\varphi\colon A\to B$ is the structure map.

  Now this morphism~$\overline{\varphi}$ is unique because if~$\psi$ would be another morphism extending~$\mathrm{i}\colon A\to S^{-1}A$ we'd find
  \begin{equation}
    \psi\left( \frac{r}{s} \right)=\psi\left( \frac{r}{1} \right)\psi\left( \frac{1}{s} \right)=\varphi(r)\varphi(s)^{-1}
  \end{equation}
  as we split the fraction into parts on which~$\mathrm{i}$ works.
\end{exercise}

\begin{exercise} % TODO this might be a little short
  The definition is already given in the hint. The checks for the~$S^{-1}A$\nobreakdash-module structure are trivial and the universal property is satisfied by the proof from~\autoref{exercise:23d}.
\end{exercise}

\begin{exercise}
  The isomorphism is given by
  \begin{equation}
    \frac{1}{s}\left( m_1,\dots,m_n \right)\mapsto \left( \frac{m_1}{s},\frac{m_2}{s},\dots,\frac{m_n}{s} \right)
  \end{equation}
  with the inverse map being
  \begin{equation}
    \left( \frac{m_1}{s_1},\ldots,\frac{m_n}{s_n} \right)\mapsto\frac{1}{\prod_{i=1}^n s_i}\left( m_1\prod_{i\neq 1}s_i,\ldots,m_n\prod_{i\neq n}s_i \right).
  \end{equation}

  In the infinite case the product of the nominators is not defined. In the scenario of the hint that is given, the image under the inverse map has both a division by zero and a multiplication by infinity.
\end{exercise}

\begin{exercise}
  It is possible to proof that
  \begin{equation}
    \mathbb{Z}/(m)\otimes_\mathbb{Z}\mathbb{Z}/(n)\cong\mathbb{Z}/(d)
  \end{equation}
  with~$d\colonequals\gcd(m,n)$. In order to do so: observe that
  \begin{equation}
    x\otimes y=\left( xy \right)\otimes 1=xy\left( 1\otimes 1 \right)
  \end{equation}
  hence~$1\otimes 1$ is the generator of a cyclic group that represents the tensor product. Now~$d(1\otimes 1)=0$ because both~$m(1\otimes 1)$ and~$n(1\otimes 1)$ are zero by bringing it into the correct factor and using B\'ezout's identity. So the order of the cyclic group divides~$d$.
  
  Now we can map the direct product into~$\mathbb{Z}/(d)$ in an obvious way, which induces a map from the tensor product to~$\mathbb{Z}/(d)$. The element~$1\otimes 1$ is mapped to~$1$ and consequently has order~$d$, so there is an element of order \emph{at least~$d$}. Hence we have obtained the desired isomorphism.

  In this special case:~$\gcd(10,12)=2$.
\end{exercise}

\begin{exercise} % TODO expand / rewrite
  That~$f\otimes\identity\colon M\otimes N\to M''\otimes N$ is still surjective is obvious.

  For the surjection of~$M'\otimes N$ onto the kernel of~$f\otimes\identity$ we have to prove that in
  \begin{equation}
    M\otimes N\to (M\otimes N)/\Image((M'\to M)\otimes\identity)\to M''\otimes N,
  \end{equation}
  the last morphism being the induced morphism from~$f\otimes\identity$ is invertible. Construct the induced~$\overline{\varphi}$ from
  \begin{equation}
    \map{\varphi}{M''\otimes N}{(M\otimes N)/\Image((M'\to M)\otimes\identity)}{m''\otimes n}{m\otimes n+\Image((M'\to M)\otimes\identity)}
  \end{equation}
  where~$m\in f^{-1}(m'')$ by surjectivity of~$f$. Now~$\varphi$ is well defined, bilinear and therefore extends to~$\overline{\varphi}$. Now the composition~$\overline{\varphi}\circ\overline{f\otimes\identity}$ is the identity, hence the induced morphism is invertible and we have obtained exactness.
\end{exercise}


\chapter{Sheaves}
\section{Motivating example: the sheaf of differentiable functions}

\begin{exercise}
  As every element of~$\mathcal{O}_p\setminus\mathfrak{m}$ is nonzero in a neighbourhood of~$p$ we can restrict an element such that it is invertible there, a property which is preserved when taking the stalk. Hence the germ of a non-vanishing function is invertible and~$\mathfrak{m}$ is invertible.
\end{exercise}

\begin{exercise} % TODO find out what is asked for
  I don't really have a differential geometry background and I fail to see what should be proved. But I should revisit this exercise later.
\end{exercise}


\section{Definition of sheaf and presheaf}

\begin{exercise}
  A functor assigns an object in~$\category{Sets}$ to every object in the category of open sets of the topological space~$X$. As an inclusion of sets is reflected as a morphism of the two open sets involved, this is translated to a morphism of sets in the codomain category\footnote{I have never seen this terminology, don't shoot me if I missed some more obvious wording.}. Identities are preserved by functors, so~$\res_{U,U}=\identity_{\mathcal{F}(U)}$ by definition. For the commutativity of restriction maps, this is by the \emph{contravariance} of the functor.
\end{exercise}

\begin{exercise} % TODO finish this
  \begin{enumerate}
    \item The presheaf axioms are (trivially) true by restriction of functions. Yet it is not a sheaf, take~$x\mapsto|x|$. On~$\ball(0,n)$ the function is bounded by~$n$, we can take~$\mathbb{C}=\bigcup_{n\in\mathbb{N}}\ball(0,n)$ and glue together a function on all of~$\mathbb{C}$ yet it is not bounded.

    \item For some reason I am struggling with wording a good answer, yet it boils down to branch points and some open set that is not simply connected.
  \end{enumerate}
\end{exercise}

\begin{exercise}
  As we have maps~$\mathcal{F}(\bigcup U_i)\to\mathcal{F}(U_i)$ (or arbitrary unions of~$U_i$) this is a limit: the arrows from our desired object map \emph{to} all the objects as described in~2.4.4.
\end{exercise}

\begin{exercise}
  \begin{enumerate}
    \item Such functions are defined in their points, if all the restrictions agree for a covering, the functions are obviously equal. The identity axiom is satisfied. But given a covering on which all restrictions agree we can just paste together a global function, hence the gluability axiom is satisfied~too.

      Glueing functions preserves their extra properties: these are all defined in a neighbourhood of a point, hence valid in an open set and therefore lift to the global function.

    \item Analogous.
  \end{enumerate}
\end{exercise}

\begin{exercise}
  The presheaf axioms are trivially satisfied, local functions restrict easily. The identity axiom is obvious too. For the gluability: observe that the compatibility boils down to defining a function on the \emph{connected components} of the covering and therefore is satisfied too.
\end{exercise}

\begin{exercise} % TODO do this
  I should write a proper answer to this exercise, but I don't feel like it at the moment so for numbering purposes, there is this placeholder.
\end{exercise}

\begin{exercise} % TODO do this
  I should write a proper answer to this exercise, but I don't feel like it at the moment so for numbering purposes, there is this placeholder.
\end{exercise}

\begin{exercise}
  The presheaf axioms are satisfied because~$f$ is continuous hence the inverse~$f^{-1}$ maps open sets to open sets. The presheaf axioms are satisfied on the open sets in the domain of the map and the presheaf axioms for~$f_*\mathcal{F}$ are a subset (sketchy wording) of those for~$\mathcal{F}$.

  The sheaf axioms are satisfied because inverse maps are compatible with unions and intersections. As for the presheaf axioms: everything is transferred.
\end{exercise}

\begin{exercise}
  Using the definition of the direct limit and the construction as described in~\autoref{exercise:23c} we see that \emph{less} relations have to be quotiented, as the direct system defining~$(f_*\mathcal{F})_y$ is contained in the direct system defining~$\mathcal{F}_p$.
\end{exercise}

\begin{exercise}
  As germs in the stalk~$\mathcal{F}_x$ are equivalence classes of functions defined in the neighbourhood modulo equality on a (smaller) neighbourhood, we can define the~$\mathcal{O}_{X,x}$\nobreakdash-module structure on~$\mathcal{F}_x$ by defining it on representatives of these classes. By commutativity of~(3.2.12.1) this is well defined. The actual checking of the structure is straight-forward and familiar.
\end{exercise}



\part{Schemes}

\chapter{Toward affine schemes}
\section{Toward schemes}

\begin{exercise} % TODO do this
  I should be ashamed of myself, not being able to answer this question.
\end{exercise}

\begin{exercise} % TODO do this
  I should be ashamed of myself, not being able to answer this question.
\end{exercise}


\section{The underlying set of affine schemes}

\begin{exercise}
  \label{exercise:42a}
  \begin{enumerate}
    \item\label{enumerate:42a-a} So we're looking for the prime ideals of~$\Spec k[\epsilon]/(\epsilon^2)$. But these correspond to the prime ideals of~$\Spec k[\epsilon]$ containing~$(\epsilon^2)$. Now the only prime ideal of this form is~$(\epsilon)$. This corresponds to the polynomials in~$\epsilon$ with no constant term. If there would be a constant term, \ie, something of the form~$a+b\epsilon$ it would be invertible modulo~$\epsilon^2$ using a geometric series. There is only one point.

      Notice that~$\Spec k[\epsilon]/(\epsilon)$ is not an integral domain:~$\epsilon^2$ is contained in~$(0)$ yet~$\epsilon\notin(0)$.

    \item\label{enumerate:42a-b} By commutative algebra the prime ideals of the localization correspond to the prime ideals of~$\Spec k[x]$ not containing~$(x)$. So the set~$\Spec k[x]_{(x)}$ corresponds to~$\Spec k[x]\setminus\left\{ (x) \right\}$ because there is only one prime ideal containing~$x$ namely~$(x)$: if there would be another one we could reduce it to a constant ending up the whole ring, a contradiction.
  \end{enumerate}
\end{exercise}

\begin{exercise}
  Using the discriminant we obtain the two roots of the quadratic which look like
  \begin{equation}
    x_{1,2}=\frac{-a\pm\sqrt{a^2-4b}}{2}
  \end{equation}
  where~$a^2-4b<0$. Now using operations of~$\mathbb{R}$ we can reduce this to~$i$.
\end{exercise}

\begin{exercise}
  This set corresponds to all polynomials that are irreducible over~$\mathbb{Q}$. There are the obvious points~$(x-a)$ where~$a\in\mathbb{Q}$, but all roots of polynomials are present too but they are glued together by the corresponding Galois actions. It corresponds to the identification of roots in the algebraic closure~$\mathbb{Q}^{\alg}$.
\end{exercise}

\begin{exercise}
  Suppose~$\mathfrak{p}$ is a prime ideal that is not a principal ideal. Take two essential generators~$f(x,y)$ and~$g(x,y)$ (\ie, with not all factors of one contained in the other). This must be possible because otherwise we wouldn't have a principal ideal: one can be written as a product of the other with a polynomial containing the missing factors. Now because~$\mathfrak{p}$ is prime we can remove all common factors.

  By applying the Euclidean algorithm in~$\mathcal{C}(x)[y]$ we can find a polynomial in the variable~$x$ contained in~$(f(x,y),g(x,y))\subseteq\mathfrak{p}$, which by the algebraic closedness of~$\mathcal{C}$ reduces to a linear factor~$(x-a)$ contained in~$\mathfrak{p}$ and analogously~$(y-b)\in\mathfrak{p}$.

  Obviously any principal ideal must be generated by an irreducible polynomial. So having reduced all non-principal ideals to ideals of the form~$(x-a,y-b)$ we have finished the proof.
\end{exercise}

\begin{exercise}
  The first maximal ideal is~$(x^2+y^2-4,x-y)$ while the second is~$(x^2+y^2-4,x+y)$. The residue fields are~$\mathbb{Q}(\sqrt{2})$ in both cases: substituting the second generator in the first yields this result.
\end{exercise}

\begin{exercise} % TODO do this
  I'm still wondering whether my answer is correct.
\end{exercise}

\begin{exercise}
  \label{exercise:42g}
  I have used this fact in~\autoref{exercise:42a}\ref{enumerate:42a-a}. It boils down to
  \begin{equation}
    A/J\cong (A/I)/(J/I) 
  \end{equation}
  where~$I\subseteq J$ are prime ideals of~$A$, and this is equivalent to~$\overline{J}$ being prime in~$A/I$.
\end{exercise}

\begin{exercise} % TODO rewrite this
  \label{exercise:42h}
  I have used this fact in~\autoref{exercise:42a}\ref{enumerate:42a-b}. A prime ideal of~$A$ that contains an element of~$S$ will the whole become~$S^{-1}A$ under localization. A prime ideal of~$A$ disjoint of~$S$ remains prime because the product~$(p_1/s_1)(p_2/s_2)$ is in the prime ideal if and only if~$(p_1p_2)/(s_1s_2)$ is in the prime ideal, but we can multiply with~$s_1s_2$ as this is by multiplicativity of~$S$ not a member of the prime ideal. Now we have reduced it to~$p_1p_2$ in the prime ideal.
\end{exercise}

\begin{exercise} % TODO do this
  I haven't figured it out yet.
\end{exercise}

\begin{exercise}
  Assume~$b_1b_2\in\phi^{-1}(\mathfrak{p})$, we have
  \begin{equation}
    \phi(b_1b_2)=\phi(b_1)\phi(b_2)\in\phi\left( \phi^{-1}(\mathfrak{p}) \right)=\mathfrak{p},
  \end{equation}
  hence~$\phi(b_1)$ or~$\phi(b_2)$ as~$\phi(\phi^{-1}(\mathfrak{p}))=\mathfrak{p}$ is a prime ideal in~$A$. We obtain that~$\phi^{-1}(\phi(b_1))=b_1$ or~$\phi^{-1}(\phi(b_2))=b_2$ must be an element of~$\phi^{-1}(\mathfrak{p})$.
\end{exercise}

\begin{exercise}
  \label{exercise:42k}
  \begin{enumerate}
    \item\label{enumerate:42k-a} Using~\autoref{exercise:42g} everything is already clear: the primes of~$A$ containing~$I$ form a subset of~$\Spec A$ and~$\phi^{-1}$ is an inclusion-preserving bijection, giving us the suggested picture.

    \item Using~\autoref{exercise:42h} everything is analogous.
  \end{enumerate}
\end{exercise}

\begin{exercise}
  The fiber of~$a\in\mathbb{C}$ corresponds to the preimage of the prime ideal (in this case: maximal ideal) defining~$a$, \ie, $(x-a)$. This obviously gives us~$y^2-a=(y-\sqrt{a})(y+\sqrt{a})$, hence the result.
\end{exercise}

\begin{exercise} % TODO is it?
  \begin{enumerate}
    \item This is a restatement of~\autoref{exercise:42k}\ref{enumerate:42k-a}.

    \item The Nullstellensatz gives us that all maximal ideals (\ie, points) of~$\mathbb{C}^n$ are exactly the ideals of the form~$(x_1-a_1,\ldots,x_m-a_m)$, which by~$\phi$ are mapped to the corresponding points of~$\mathbb{C}^n$.
  \end{enumerate}
\end{exercise}

\begin{exercise} % TODO fix this
  In the notation of~\autoref{exercise:42k}\ref{enumerate:42k-a} we have~$I=(x_1,\ldots,x_n)$ and~$B$ being~$\mathbb{Z}[x_1,\ldots,x_n]$. A point of~$\mathbb{A}_{\mathbb{F}_p}^n$ corresponds to a polynomial in~$n$~variables that is irreducible over~$\mathbb{F}_p$, which lies in the fiber over~$(p)$ because it is contained in the image of the induced ring map.
\end{exercise}

\begin{exercise}
  \begin{enumerate}
    \item We have that every prime ideal contains all nilpotents: if~$c$ is a nilpotent such that~$c^n=0$, we immediately find that~$c$ is an element of the prime ideal. The bijection is between primes of~$A/I$ and primes of~$A$ containing~$I$, but this latter set contains all primes, hence there is a bijection of the underlying sets.

    \item Let's check the axioms. The sum of two nilpotents is again a nilpotent: take~$x$ and~$y$ nilpotents such that~$x^n=y^m=0$, we easily obtain
      \begin{equation}
        (x+y)^{n+m}=\sum_{i=0}^{n+m}\binom{n+m}{i}x^iy^{n+m-i}
      \end{equation}
      such that there always is a vanishing factor present in the expansion. Closed under multiplication is obviously true too: we have~$(bx)^n=b^nx^n=0$ for~$b\in B$.
  \end{enumerate}
\end{exercise}

\begin{exercise} % TODO copy
  I have seen this exercise in my Commutative algebra course. It might be a good idea to state it here as well.
\end{exercise}

\begin{exercise} % TODO do this
  I fail to find a decent argument.
\end{exercise}

\begin{exercise}
  A polynomial~$f\in k[x]$ corresponds to~$\sum_{k=0}^na_kx^k$. Now considering it over~$k[x,\epsilon]/(\epsilon^2)$ and ``evaluating'' it at~$x+\epsilon$ we find
  \begin{equation}
    f(x+\epsilon)=\sum_{k=0}^na_k(x+\epsilon)^k=\sum_{k=0}^na_k\left( x^n+nx^{n-1}\epsilon \right)
  \end{equation}
  because every term containing~$\epsilon^2$ is gone. If we move the first term of the inner sum to the left-hand side and dividing both sides by~$\epsilon$ (which isn't really possible, but for the sake of argument we can assume this), we see the fact~$(x^n)'=nx^{n-1}$.
\end{exercise}


\section{Visualing schemes I: generic points}

There are no exercises in this section.


\chapter{The structure sheaf and the definition of schemes in general}
\section{The structure sheaf of an affine scheme}

\begin{exercise}
  We have~$\vanishing(f)=\left\{ [\mathfrak{p}]\in\Spec A\,|\,f\in\mathfrak{p} \right\}$, functions not vanishing outside~$\vanishing(f)$ are by~\autoref{exercise:45e} correspond to~$g$ such that~$f^n\in(g)$. Therefore the localisation~$\mathcal{O}_{\Spec A}(\distinguished(f))$ consists of fractions such that the denominator is an element of the principal ideal~$(g)$ such that~$f^n\in(g)$. Remark that this is obviously a multiplicative set and the isomorphism is given by the result of~\autoref{exercise:45e}.
\end{exercise}

\begin{exercise}
  This really boils down to judiciously replacing~$A$ by~$A_f$. We obtain:
  \begin{quote}
    We check identity on the base. Suppose that~$\Spec A_f=\bigcup_{i\in I}\distinguished(f_i)$ where~$i$ runs over some index set~$I$. Then there is some finite subset of~$I$, which we name~$\left\{ 1,\ldots,n \right\}$, such that~$\Spec A_f=\bigcup_{i=1}^n\distinguished(f_i)$, \ie,~$(f_1,\ldots,f_n)=A_f$ (quasicompactness of~$\Spec A_f$, \autoref{exercise:45c}).

    Suppose we are given~$s\in A_f$ such that~$\res_{\Spec A_f,\distinguished(f_i)}s=0$ in~$A_{f_i}$ for all~$i$. We wish to show that~$s=0$. The fact that~$\res_{\Spec A_f,\distinguished(f_i)}s=0$ in~$A_{f_i}$ implies that there is some~$m$ such that for each~$i\in\left\{ 1,\ldots,n \right\}$, ~$f_i^ms=0$. Now~$(f_1^m,\ldots,f_n^m)=A_f$, for example, from
    \begin{equation}
      \Spec A_f=\bigcup_{i=1}^n\distinguished(f_i)=\bigcup_{i=1}^n\distinguished(f_i^m),
    \end{equation}
    so there are~$r_i\in A_f$ with~$\sum_{i=1}^nr_if_i^m=1$ in~$A_f$, from which
    \begin{equation}
      s=\left( \sum_{i=1}^nr_if_i^m \right)s=\sum_{i=1}^nr_i(f_i^ms)=0.
    \end{equation}
    Thus we have checked the ``base identity'' axiom for~$\Spec A_f$.
  \end{quote}
\end{exercise}

\begin{exercise}
  \label{exercise:51c}
  Again, replacing~$A$ with~$A_f$, open covers of~$\Spec A$ with open covers of the open subspace~$\Spec A_f$ and copying the entire proof suffices. Only three occurences of~$A$ should be replaced with~$A_f$
\end{exercise}

\begin{exercise}
  \label{exercise:51d}
  We have to redo Theorem~5.1.2, but now for this more general construction\todo{\autoref{exercise:51d}: prove that~$\tilde{M}$ is a sheaf}.
\end{exercise}


\section{Visualing schemes II: nilpotents}

There are no exercises in this section.


\section{Definition of schemes}

\begin{exercise}
  If~$f\colon A'\to A$ is a isomorphism of rings, the induced affine schemes are obviously isomorphic too: the underlying spaces are homeomorphic (the vanishing sets are ``equal'' by the isomorphism) and we can put the induced isomorphism for every localization that occurs in the corresponding sheaves.

  If~$f\colon\Spec A\to\Spec A'$ is an isomorphism of affine schemes, the rings of global sections are isomorphic. Now this is a bijection because the rings~$A$ and~$A'$ determine everything there is to know about these ringed spaces and their isomorphisms.
\end{exercise}

\begin{exercise}
  We have the homeomorphism between~$\distinguished(f)$ and~$\Spec A_f$ by Section~4.5. Now the sheaves are isomorphic too: the restriction~$\mathcal{O}_{\Spec A}|_{\distinguished(f)}$ has~$\sections(\distinguished(f),\mathcal{O}_{\Spec A})=A_f$ as global ring of sections which induces the desired isomorphism.
\end{exercise}

\begin{exercise}
  \label{exercise:53c}
  Consider~$p\in U\subseteq X$. Take~$V$ a neighbourhood of~$p$ such that~$(V,\mathcal{O}_X|_V)$ is an affine scheme, \ie, is isomorphic to~$(\Spec A,\mathcal{O}_{\Spec A})$ for some ring~$A$. Now consider the restriction of~$\mathcal{O}_{\Spec A}$ to~$U\cap V$ which is again an affine scheme\todo{\autoref{exercise:53c}: an open subscheme of an affine scheme is again affine, but formalize this please}.
\end{exercise}

\begin{exercise}
  By definition of a scheme we have for~$U\subseteq X$ open that~$(U,\mathcal{O}_X|_U)$ is an affine scheme, which induces the Zariski topology on~$U$. Now all the restrictions are given from that point, so we just have to take arbitrary unions of open sets regarding every open~$U$ to obtain all open sets in~$X$.
\end{exercise}

\begin{exercise}
  \begin{enumerate}
    \item Everything follows from~\autoref{exercise:46t}: we have the homeomorphism~$\coprod_{i=1}^n\Spec A_i\to\Spec\prod_{i=1}^nA_i$ which gives us that the finite disjoint union is isomorphic to the spectrum of a finite product of rings, which is an affine scheme, just take~$A=\prod_{i=1}^nA_i$ in the definition of the affine scheme~$\coprod_{i=1}^n\Spec A_i$.

    \item By~\autoref{exercise:46d}\ref{enumerate:46d-a} we have that affine schemes are quasicompact, but the infinite disjoint union can be covered by taking the (affine) opens~$\Spec A_i$, which are by definition disjoint but open. We've obtained an infinite cover that cannot be reduced.\qedhere
  \end{enumerate}
\end{exercise}

\begin{exercise}
  The stalk in~$[\mathfrak{p}]$ is defined as the direct limit of the sections over all open subsets containing~$[\mathfrak{p}]$. These open sets are generated by the~$\distinguished(f)$ for~$f\notin\mathfrak{p}$. The sections over these sets are the localizations~$A_f$. In the direct system these will be identified accordingly, but because~$\mathfrak{p}$ is prime, one will never obtain a localization at an element of~$\mathfrak{p}$. Yet all elements in~$A\setminus\mathfrak{p}$ \emph{will} be inverted.
  
  The localization at a prime ideal on the other hand is the localization at the complement~$A\setminus\mathfrak{p}$. We obtain the desired result.
\end{exercise}


\chapter{Some properties of schemes}
\input{ch6.tex}

\part{Morphisms of schemes}

\part{Quasicoherent sheaves}

\part{More}
\end{document}
